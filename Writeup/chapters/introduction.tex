%!TEX root = ../thesis.tex
\chapter{Introduction}
\label{introduction}

Over the past decade, the increasing convenience and power of smartphones (e.g. Apple iPhones or Android-powered phones) and mobile wearables (e.g. Fitbit Alta, Apple Watch, Jawbone UP) have induced furious adoption by the public.  These devices come packed with data collecting sensors able to detect a wide swath of information, ranging from automatic collection of location data and physical activity data to user check-ins on dietary data or mental state-of-mind.  While this data is usually collected for the individuals to track their own lifestyles or for intellectual endeavors of the companies making the devices, there are a large number of life-changing applications \citep{Ramkumar2017}.  

While classical healthcare models have previously been composed of intermittent trips to the doctor's office, sensors on these mobile devices allow for continuous monitoring of health indicators that greatly enhance long-term conditions, including heart disease, arthritis, or diabetes.  As such, the healthcare industry has started to use indicators of health, such as physical activity or mental state-of-mind to adopt methods in utilizing patient data for personalizing and enhancing individual treatments.  This use of mobile devices in healthcare is called Mobile Health, or mHealth for short.

One promising mHealth application is to lower heart disease risks by using the frequent contact that individuals have with their mobile devices to encourage physical activity.  From a naive but well-intentioned researcher's perspective, it may seem that the optimal course of action is to frequently prompt patients to engage in physical activity; however, from a behavioral standpoint, it is often times not ideal or even feasible for patients to engage in physical activity, such as when a patient discovers it is raining outside, has just returned from a long walk, is occupied with work, or even is in the middle of commuting.  Furthermore, inundating patients with suggestions of physical activity when it is not possible may cause disengagement due to overburdening, disinterest, or distrust of the application.  Such a mHealth application must be carefully tuned to give well-timed suggestions that ``intervene'' with a patient's ordinary day.  

HeartSteps is an academically developed mobile application that aims to address the given problem: optimally delivering small but precisely-timed personalized interventions to maintain physical activity.  While it is possible to cater to the average user, HeartSteps must factor in both the contexts in which it is delivering interventions, as well as information about individual patients' behavioral responses. This type of mHealth application is known as ``Just-In-Time Adaptive Interventions,'' and not only has the benefit of minimizing the invasiveness of each suggestion (compare several suggestions a day of standing up for a minute each versus a broad suggestion to a broad suggestion each day of walking several miles), but also lends itself to statistical analysis with its plentiful data.  

In this thesis, we apply variants of a rapidly-learning reinforcement learning algorithm called the Thompson Sampler to the HeartSteps mHealth problem.  We experimentally investigate and evaluate the impact of personalizing individual patient's treatments to the overall efficacy of each variant algorithm using real trial data, and use these results as guidelines to craft the next iteration of HeartSteps.

% Mobile Health just starting, but lots of challenges 
% Challenge: How to make Interventions


